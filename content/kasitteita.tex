\section{Peruskäsitteitä}
\subsection{Määritelmiä}

On hyvä huomioida ero termien välillä kun puhutaan staattisesta ja
%                                       "heikko" = "löyhä"?
dynaamisesta, tai toisaalta vahvasta ja heikosta tyypittämisestä. Myös
dynaamisesti tyypitetty kieli voi nostaa virheen väärien tietotyyppien
käyttämisestä ohjelman suorittamisen aikana, etenkin jos se on vahvasti
tyypitetty kieli. Tässä tutkielmassa keskitytään kuitenkin lähinnä
staattiseen ja dynaamiseen tyypitykseen, eli siihen miten tietotyyppeihin
liittyviä virheitä käsitellään käännös aikana, jo ennen ohjelman
suorittamista.