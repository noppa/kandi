\chapter{Peruskäsitteitä}

\subsection{Tyyppijärjestelmien luokitteleminen} Ohjelmointikielten
tyyppijärjestelmien jakaminen staattisesti ja dynaamisesti
tyyppitarkastettuihin perustuu ohjelman kehitysvaiheeseen jossa tarkastaminen
tapahtuu. Staattisella tyyppitarkastamisella viitataan ohjelman tyyppien
analyysiin ennen ohjelman suorittamista, esimerkiksi käännösaikana, kun taas
dynaaminen tyyppitarkastus varmistaa arvojen tyyppien oikeellisuuden ohjelmaa
suoritettaessa. Tyyppijärjestelmät voidaan jaotella myös muiden
ominaisuuksien perusteella, esimerkiksi vahvoihin ja heikkoihin
tyyppijärjestelmiin. Näiden termien merkitys ei ole tarkasti määritelty,
mutta yleisesti niillä viitataan tapaan jolla kieli käsittelee määritelmästä
eriävät tai virheelliset tyypit\cite{CornellTransitionToOO}. Vahvasti
tyypitetyssä kielessä tällainen aiheuttaisi yleensä virheen, heikosti
tyypitetyssä kielessä tyypeille voitaisiin tehdä implisiittisiä
tyyppimuunnoksia tyyppien yhteensopivuuden saavuttamiseksi.

JavaScript on dynaamisesti tarkastettu, heikosti tyypitetty kieli.
Esimerkiksi ohjelma \inlinecode{\dblquoted{teksti}.potenssiin(3)} antaa
staattisesti tyyppitarkastetussa kielessä virheen jo käännösaikana, mikäli
metodia \inlinecode{potenssiin} ei ole tekstityyppisille arvoille määritetty.
JavaScriptiä suorittava ympäristö sen sijaan hyväksyisi ohjelman ja sallisi
sen suorittamisen. Virhe olemattoman metodin kutsumisesta ilmenisi vasta jos
ohjelman suoritus evaluoi kyseisen ilmaisun. Lisäksi esimerkiksi ilmaisu
\inlinecode{\dblquoted{teksti} + 2} ei aiheuttaisi virhettä edes
suoritusaikana, sillä heikoille tyyppijärjestelmille ominaisesti JavaScript
muuttaisi numeron 2 string-muotoon ennen summausoperaation arviointia. Tässä
tutkielmassa keskitytään lähinnä JavaScriptin tyyppien staattiseen ja
dynaamiseen, eli käytännössä käännös- ja ajonaikaiseen tarkastamiseen. Jotkin
esitellyistä työkaluista myös tiukentavat kielen sallimia operaatioita siten,
että esimerkiksi yllä esitettyä \inlinecode{\dblquoted{teksti} + 2} ohjelmaa
ei enää sallittaisi. Monia muita heikoille tyyppijärjestelmille
tavallisia ominaisuuksia jää kuitenkin tarkistamatta.
