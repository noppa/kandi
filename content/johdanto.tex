\chapter{Johdanto} \label{Johdanto}

Ohjelmointi on pohjimmiltaan tietorakenteiden käsittelyä ja yksi tärkeimmistä
tietorakenteen ominaisuuksista on sen tyyppi. Muuttuja \dblquoted{nimi} voi olla
datatyypiltään teksti ja muuttuja \dblquoted{ikä} voi olla numero, eikä näitä kahta voi
huolettomasti sekoittaa. Ohjelman tila ei ole järkevä jos se sanoo henkilön
iän olevan \dblquoted{Matti}. Se miten ohjelmointikielissä käsitellään arvojen tyyppejä
vaihtelee kuitenkin suuresti.

Tässä tutkielmassa käsitellään JavaScriptiä, sekä kolmea työkalua jotka
rakentavat staattisesti tarkastettavan tyyppijärjestelmän
JavaScriptin päälle. JavaScript on \textit{dynaamisesti tyypitetty} ohjelmointikieli,
jonka alkuperäinen käyttötarkoitus oli lisätä verkkosivuille pieniä
interaktiivisia ominaisuuksia, kuten lomakkeiden validointia. JavaScriptillä
toteutettavien ohjelmien koko, monimutkaisuus ja tärkeys on kuitenkin viime
vuosien aikana kasvanut alkuperäistä tarkoitusperää suuremmaksi, kun sillä on
alettu toteuttaa esimerkiksi kartta-, kirjoitus- ja hallintapalveluita jotka
toimivat selaimessa, siten ettei käyttäjän tarvitse asentaa erillistä
tietokoneohjelmaa palvelun käyttöön. JavaScriptia on myös alettu käyttää
verkkosivujen \textit{front-end} käyt\-tö\-liit\-ty\-mä\-to\-teu\-tuk\-sen
lisäksi tietokone- ja kän\-nyk\-kä\-so\-vel\-luk\-sis\-sa, sekä
pal\-ve\-lin\-kie\-le\-nä \textit{back-endillä}.

Tutkielmassa esitellään TypeScript, Flow ja Closure-kääntäjä, joista jokainen on
tarkoitettu työkaluksi sellaisten ohjelmien kehittämiseen, jotka muuten kehitettäisiin
JavaScriptilla. Päämääränä on esitellä kuinka dynaamisesti tyypitetty JavaScript voidaan
muuttaa staattisesti tyyppitarkastetuksi TypeScriptin, Flown, tai Closure-kääntäjän
avulla, sekä mitä hyötyä siitä voi olla. Työkalujen hyötyjä ja haittoja
vertaillaan sekä toisiinsa että tavalliseen JavaScript-koodiin.
