\chapter{Johdanto} \label{Johdanto}

Ohjelmointi on pohjimmiltaan tietorakenteiden käsittelyä ja yksi tärkeimmistä
tietorakenteen ominaisuuksista on sen tyyppi. Muuttuja “nimi” voi olla
datatyypiltään teksti ja muuttuja “ikä” voi olla numero, eikä näitä kahta voi
huolettomasti sekoittaa. Ohjelman tila ei ole järkevä jos se sanoo henkilön
iän olevan “Matti”. Se miten ohjelmointikielissä käsitellään arvojen tyyppejä
vaihtelee kuitenkin suuresti.

Tässä tutkielmassa käsitellään JavaScriptiä, sekä kolmea työkalua jotka
lisäävät rakentavat staattisesti tarkastettavan tyyppijärjestelmän
JavaScriptin päälle. JavaScript on dynaamisesti tyypitetty ohjelmointikieli,
jonka alkuperäinen käyttötarkoitus oli lisätä verkkosivuille pieniä
interaktiivisia ominaisuuksia, kuten lomakkeiden validointia. JavaScriptillä
toteutettavien ohjelmien koko, monimutkaisuus ja tärkeys on kuitenkin viime
vuosien aikana kasvanut alkuperäistä tarkoitusperää suuremmaksi, kun sillä on
alettu toteuttaa esimerkiksi kartta-, kirjoitus- ja hallintapalveluita jotka
toimivat selaimessa, siten ettei käyttäjän tarvitse asentaa erillistä
ohjelmaa palvelun käyttöön.

Tutkielmassa esitellään TypeScript, Flow ja Closure-kääntäjä, joista jokainen on
kehitetty työkaluksi parempien JavaScript-ohjelmien kehittämiseksi.
Tarkastelussa ilmenee, että staattinen tyypitys voi nopeuttaa ohjelman
kehittämistä, vähentää ohjelmoijan tekemien virheiden määrää ja parantaa
valmiin ohjelman tehokkuutta. Toisaalta nähdään myös, että valinta staattisen
ja dynaamisen tyypityksen välillä sisältää kompromisseja, etenkin kun
tyyppijärjestelmä on erillinen työkalu eikä kieleen alusta asti kehitetty
ominaisuus.
