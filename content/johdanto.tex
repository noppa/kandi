\chapter{Johdanto} \label{Johdanto}

Ohjelmointi on pohjimmiltaan tietorakenteiden käsittelyä ja yksi tärkeimmistä
tietorakenteen ominaisuuksista on sen tyyppi. Muuttuja \dblquoted{nimi} voi olla
datatyypiltään teksti ja muuttuja \dblquoted{ikä} voi olla numero, eikä näitä kahta voi
huolettomasti sekoittaa. Ohjelman tila ei olisi järkevä jos henkilön
iäksi sallittaisiin \dblquoted{Matti}.
iän olevan \dblquoted{Matti}. Se miten ohjelmointikielissä käsitellään arvojen tyyppejä
vaihtelee kuitenkin suuresti.

Tässä tutkielmassa käsitellään JavaScript-ohjelmointikieltä,
sekä kolmea työkalua\newline
jotka rakentavat staattisesti tarkastettavan tyyppijärjestelmän
JavaScriptin päälle.\newline
JavaScriptin alkuperäinen käyttötarkoitus
oli lisätä verkkosivuille pieniä interaktiivisia ominaisuuksia,
kuten lomakkeiden validointia. JavaScriptillä
toteutettavien ohjelmien koko, monimutkaisuus ja tärkeys on kuitenkin viime
vuosien aikana kasvanut alkuperäistä tarkoitusperää suuremmaksi, kun sillä on
alettu toteuttaa esimerkiksi kartta-, kirjoitus- ja hallintapalveluita jotka
toimivat selaimessa, siten ettei käyttäjän tarvitse asentaa erillistä
tietokoneohjelmaa palvelun käyttöön.
JavaScriptin käyttö on levinnyt verkkosivujen asiakaspuolen käyttöliittymän
toteutuksesta tietokone- ja älypuhelinsovelluksiin sekä verkon
asiakas-palvelin-arkkitehtuurissa myös palvelinsovelluksien kieleksi.

Tutkielmassa esitellään TypeScript, Flow ja Closure-kääntäjä, joista jokainen on
tarkoitettu työkaluksi sellaisten ohjelmien kehittämiseen,
jotka muuten kehitettäisiin\newline
JavaScriptillä. Päämääränä on tarkastella kuinka \textit{dynaamisesti tyypitetty}
JavaScript voidaan täydentää staattisesti tyyppitarkastetuksi
TypeScriptin, Flown, tai Closure-kääntäjän avulla, sekä mitä hyötyä siitä voi
olla. Työkalujen hyötyjä ja haittoja vertaillaan sekä toisiinsa että
tavalliseen JavaScript-koodiin.

Vuosittainen JavaScript-yhteisölle
suunnattu \textit{State Of JS} kyselytutkimus kartoittaa kehittäjien
kiinnostusta JavaScript-kehitykseen liittyviä työkaluja kohtaan.
Vuoden 2018 kyselyyn vastanneista 46.7\% ilmoitti
käyttäneensä TypeScriptiä ja haluavansa
käyttää sitä uudestaan \cite{StateOfJs2018}.
Vielä vuonna 2016 luku oli 21\% \cite{StateOfJs2016}. TypeScript ja
JavaScriptin staattinen tyypitys ovat siis herättäneet
JavaScript-kehittäjissä mielenkiintoa ja näyttävät kasvattavan
suosiotaan myös tulevaisuudessa.
Vuoden 2018 State Of JS kyselyyn vastanneista 33.7\% vastasi haluavansa
opetella TypeScriptiä ja 34.5\% Flow'ta.