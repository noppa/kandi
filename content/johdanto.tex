\section{Johdanto}

Kaikenlaisen ohjelmoinnin keskiössä on data ja yksi tärkeimmistä datan
ominaisuuksista on sen tyyppi. Muuttuja “nimi” voi olla datatyypiltään
teksti ja muuttuja “ikä” voi olla numero, eikä näitä kahta voi
huolettomasti sekoittaa. Ohjelman tila ei ole järkevä jos se sanoo
henkilön iän olevan “Matti”. Ohjelmointikielet voidaan hyvin karkeasti
jakaa kahteen sen mukaan miten ja missä vaiheessa niissä käsitellään
muuttujien tyyppejä.

Staattisesti tyypitetyiksi kutsutaan kieliä jotka vaativat että ohjelman
käsittelemien tietorakenteiden tyypit on tulkittavissa käännös aikana,
eli jo ennen ohjelman suorittamista. Lähes aina ohjelmointikieli saa
tämän tiedon tietotyypeistä vaatimalla koodiin erityisiä
tyyppimäärittelyitä. Ohjelman koodissa voitaisiin esimerkiksi määrittää
että henkilön ikä on aina numero. Tällöin ohjelmointikielen kääntäjään
rakennetut tarkistukset voivat vahtia jo ennen koodin suorittamista
ettei ohjelmoija virheellisesti yritä asettaa henkilön iäksi tekstiä tai
mitään muutakaan ei-numeerista arvoa. Dynaamisesti tyypitetyissä
kielissä sen sijaan vastuu oikeiden tietotyyppien käyttämisestä jätetään
ohjelmoijalle, kieli ei vaadi kehittäjältä tyyppien eksplisiittistä
määrittämistä eikä niiden oikeellisuutta tarkasteta ainakaan ennen
ohjelman suorittamista.

Kyseessä on kielen suunnittelullinen valinta. Kielen dynaaminen
tyypittäminen vaatii yleensä vähemmän varsinaisen koodin kirjoittamista
saman tuloksen saavuttamiseksi, sillä erillisiä tyyppimäärittelyitä ei
tarvitse kirjoittaa. Toisaalta staattinen tyypitys mahdollistaa monia
hyödyllisiä kehitystyökaluja ja auttaa poistamaan väärien tietotyyppien
käyttämisestä johtuvan bugien osajoukon. Kielen suunnittelijan on
löydettävä haluamansa tasapaino kielen ominaisuuksien välillä ja
arvioitava mikä on parhaaksi niihin käyttötarkoituksiin joihin kielen on
tarkoitus hyvin soveltua.