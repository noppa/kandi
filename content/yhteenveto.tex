\chapter{Yhteenveto}
Staattinen tyypitys jakaa mielipiteitä. Eksplisiittiset
tyyppimäärittelyt vaativat lisää kirjoitettavaa koodia ja liian tiukka
tyyppijärjestelmä voi rajoittaa sitä mitä kielellä voi tehdä, sillä harva
tyyppijärjestelmä on täydellinen (engl. complete).
TODO: lodash.get esimerkki tähän?

Koodin nopeasta kirjoitustahdista saatavat hyödyt jäävät kuitenkin vähäisiksi
jos lopputuotos ei toimi odotetulla tavalla. Bugit heikentävät käyttäjien
tyytyväisyyttä ohjelmaan ja voivat pahimmillaan aiheuttaa sellaista vahinkoa
että koko bugisen ohjelman julkaisusta saatu nettohyöty on negatiivinen, eli
sen käyttöönotto aiheuttaa enemmän haittaa kuin hyötyä. Staattinen tyypitys
ohjaa kirjoitettua koodia turvalliseen suuntaan kehitysvaiheen alusta
loppuun, ja torjuu tietynlaisia ohjelmointivirheitä tehokkaammin kuin
esimerkiksi ajonaikaista käyttäytymistä testaavat \textit{unit testit}.

Web-teknologioilla rakennettavat ohjelmistot ja niitä pyörittämään vaadittava
koodin määrä kasvavat kaiken aikaa isommiksi, mikä tekee uusien ohjelmoijien
sisäänotosta erityisen tärkeää. JavaScript ja muut dynaamiset kielet kuten
Python ja PHP ovat useiden aloittelijoille tarkoitettujen koulutusohjelmien
suosiossa, minkä vuoksi JavaScriptin käyttö sellaisenaan ilman esiteltyjen
kaltaisia lisätyökaluja voi houkutella enemmän hakijoiden määrää ja siten
tehdä rekrytoinnista helpompaa. Staattinen tyypitys voi kuitenkin myös
helpottaa projektiin sisään pääsyä uusille työntekijöille sen tarjoaman
dokumentaation kautta.

TODO: Pyre, Hack, Ruby

TODO: Miksi dynaamiseen kieleen staattinen tyypitys eikä suoraan
staattisesti tyypitettyä kieltä
