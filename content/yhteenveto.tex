\chapter{Yhteenveto}

Tyyppijärjestelmä on tärkeä ohjelmointikielen ominaisuus, jolla on suuri
vaikutus ohjelmistokehittäjän käyttökokemukseen.
Eksplisiittiset tyyppimäärittelyt vaativat lisää kirjoitettavaa koodia
ja liian tiukka tyyppijärjestelmä voi rajoittaa sitä mitä kielellä voi tehdä, sillä harva
tyyppijärjestelmä on täydellinen (engl. complete). 
Dynaamisesti tyypitetyn koodin nopeasta kirjoitustahdista saatavat hyödyt jäävät kuitenkin
vähäisiksi jos lopputuotos ei toimi odotetulla tavalla. Bugit heikentävät
käyttäjien tyytyväisyyttä ohjelmaan ja voivat pahimmillaan aiheuttaa
pysyvää vahinkoa saadessaan ohjelman toimimaan virheellisesti.
Staattinen tyypitys ohjaa kirjoitettua koodia turvalliseen suuntaan kehitysvaiheen alusta
loppuun ja torjuu tietynlaisia ohjelmointivirheitä tehokkaammin kuin
esimerkiksi ajonaikaista käyttäytymistä testaavat \textit{unit testit}.
Työn viidennen luvun viimeisessä esimerkissä esitelty get-kutsu on yksinkertainen
ja helppolukuinen, mutta jos muuttujan \inlinecode{object} tyyppiä myöhemmin
muutetaan toiseen muotoon muistamatta päivittää myös get-kutsua, virheellisestä
get-kutsusta voi muodostua ohjelmaan hankala bugi.

Tässä tutkielmassa on esitelty kolme työkalua, jotka lisäävät dynaamisesti
tyypitettyyn JavaScriptiin käännösaikaisen tyyppitarkastuksen sekä syntaksin
tyyppien eksplisiittiseen määrittämiseen. Flow, TypeScript ja Closure
sallivat asteittaisen siirtymisen dynaamisesti tyyppitarkastetusta staattisesti
tarkastettuun, sekä staattisen tarkastuksen ohittamisen niissä osissa koodia
joihin sitä olisi liian vaikeaa tai mahdotonta lisätä.
Samankaltaisuus ja yhteensopivuus jo ennestään suositun JavaScriptin kanssa
yhdistettynä tyyppiturvallisuuteen, koodin kirjoittamista helpottaviin
työkaluihin ja selkeämmin dokumentoituun koodiin ovat nostaneet erityisesti
TypeScriptin käytön nopeaan kasvuun. Vuosittaisessa JavaScript-yhteisölle
tarkoitetussa \textit{State Of JS} kyselytutkimuksessa 46.7\% vuonna 2018
vastanneista ilmoitti käyttäneensä TypeScriptiä ja haluavansa
käyttää sitä uudestaan \cite{StateOfJs2018}.
Vielä vuonna 2016 luku oli 21\% \cite{StateOfJs2016}. TypeScript ja
JavaScriptin staattinen tyypitys näyttävät kasvattavan suosiotaan myös
tulevaisuudessa. Vuoden 2018 State Of JS kyselyyn vastanneista 33.7\%
vastasi haluavansa\newline
opetella TypeScriptiä ja 34.5\% Flow'ta.
