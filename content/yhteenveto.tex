\chapter{Yhteenveto}

Tyyppijärjestelmä on tärkeä ohjelmointikielen ominaisuus, jolla on suuri
vaikutus ohjelmistokehittäjän käyttökokemukseen.
Liian tiukka tyyppijärjestelmä voi rajoittaa sitä mitä kielellä voi tehdä,
sillä harva tyyppijärjestelmä on täydellinen (engl. complete).
Lisäksi eksplisiittiset tyyppimäärittelyt vaativat lisää kirjoitettavaa koodia.
Funktioiden parametrien ja luokkien jäsenmuuttujien tyypit on kirjoitettava
osaksi määrittelyä silloin kun tyyppijärjestelmä ei kykene tulkitsemaan
niitä automaattisesti. Jotta parametrien ja muuttujien tyyppiannotaatioissa
voitaisiin merkata esimerkiksi objektien rakenteita, nekin voidaan
joutua kirjoittamaan auki esimerkkien \ref{lst:closure_typedef} ja
\ref{lst:flow_ts_typedef} mukaisilla tyyppikuvauksilla.
Eksplisiittisistä tyyppimäärittelyistä voi olla hyötyä
koodin dokumentaationa, mutta kehittäjä saattaa myös kokea niiden
kirjoittamisen rasitteena. Dynaamisesti tyypitettyä koodia on usein
nopeampi tuottaa ja iteroida etenkin kehitysprosessin alkuvaiheissa, jolloin
käytettyjen tyyppien rakenne voi muuttua useaan kertaan. 

Dynaamisesti tyypitetyn koodin nopeasta kirjoitustahdista saatavat hyödyt jäävät kuitenkin
vähäisiksi jos lopputuotos ei toimi odotetulla tavalla. Ohjelmistovirheet heikentävät
käyttäjien tyytyväisyyttä ohjelmaan ja voivat pahimmillaan aiheuttaa
pysyvää vahinkoa saadessaan ohjelman toimimaan virheellisesti.
Staattinen tyypitys ohjaa kirjoitettua koodia turvalliseen suuntaan kehitysvaiheen alusta
loppuun ja torjuu tietynlaisia ohjelmointivirheitä tehokkaammin kuin
esimerkiksi ajonaikaista käyttäytymistä testaavat yksikkötestit.
Työn viidennen luvun viimeisessä esimerkissä esitelty get-kutsu on yksinkertainen
ja helppolukuinen, mutta jos muuttujan \inlinecode{object} tyyppiä myöhemmin
muutetaan toiseen muotoon muistamatta päivittää myös get-kutsua, virheellisestä
get-kutsusta voi muodostua ohjelmaan hankala ohjelmistovirhe.

Tässä tutkielmassa on esitelty kolme työkalua, jotka lisäävät dynaamisesti
tyypitettyyn JavaScriptiin käännösaikaisen tyyppitarkastuksen sekä syntaksin
tyyppien eksplisiittiseen määrittämiseen. Flow, TypeScript ja Closure
sallivat asteittaisen siirtymisen dynaamisesti tyyppitarkastetusta staattisesti
tarkastettuun, sekä staattisen tarkastuksen ohittamisen niissä osissa koodia
joihin sitä olisi liian vaikeaa tai mahdotonta lisätä.
Samankaltaisuus ja yhteensopivuus jo ennestään suositun JavaScriptin kanssa
yhdistettynä tyyppiturvallisuuteen, koodin kirjoittamista helpottaviin
työkaluihin ja selkeämmin dokumentoituun koodiin ovat nostaneet erityisesti
TypeScriptin käytön nopeaan kasvuun.
