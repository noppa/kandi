\chapter{Yhteenveto}
Staattinen tyypitys jakaa mielipiteitä. Toisaalta se vaatii lisää
kirjoitettavaa koodia ja voi rajoittaa sitä mitä kielellä voi tehdä, sillä
harva tyyppijärjestelmä on täydellinen (engl. complete). Toisaalta taas
kehitystyön lopputuloksessa voidaan säästyä joiltain virheiltä ja parhaassa
tapauksessa nopea virheiden huomaaminen ja editorituki voi jopa tehdä
kehityksestä nopeampaa.

Virheiden vähentäminen staattisten tyyppien avulla kiinnostaa erityisesti
isoja yrityksiä joilla on suuria määriä yrityksen toiminnalle kriittistä
koodia dynaamisella kielellä kirjoitettuna. JavaScriptin lisäksi staattista
tyypitystä onkin kaavailtu muunmuassa Pythonin, PHP:n ja Rubyn päälle, joko
kieleen sisäänrakennettuina \dblquoted{tyyppivihjeinä} tai erillisinä
työkaluina kuten Facebookin Pyre Pythonille tai Hack PHP:lle.

Vaikka tarjolla olisi myös alun perinkin staattisesti tyypitettyjä kieliä,
joista osa on jopa käännettävissä JavaScriptiksi, monet valitsevat silti
mieluummin dynaamisesti tyypitetyn kielen päälle kehitetyn staattisen
tyyppitarkastajan. Syynä voi olla esimerkiksi olemassa olevan koodin suuri
määrä tai kielen ympärillä olevan yhteisön laajuus. Tai ehkä vaiheittainen ja
ennen kaikkea valinnainen staattinen tyypitys vain osuu sopivasti kahden
ääripään välimaastoon.
