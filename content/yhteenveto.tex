\chapter{Yhteenveto}
Staattinen tyypitys jakaa mielipiteitä. Eksplisiittiset
tyyppimäärittelyt vaativat lisää kirjoitettavaa koodia ja liian tiukka
tyyppijärjestelmä voi rajoittaa sitä mitä kielellä voi tehdä, sillä harva
tyyppijärjestelmä on täydellinen (engl. complete). 
Dynaamisen koodin nopeasta kirjoitustahdista saatavat hyödyt jäävät kuitenkin
vähäisiksi jos lopputuotos ei toimi odotetulla tavalla. Bugit heikentävät
käyttäjien tyytyväisyyttä ohjelmaan ja voivat pahimmillaan aiheuttaa
pysyvää vahinkoa saadessaan ohjelman toimimaan virheellisesti.
Staattinen tyypitys ohjaa kirjoitettua koodia turvalliseen suuntaan kehitysvaiheen alusta
loppuun, ja torjuu tietynlaisia ohjelmointivirheitä tehokkaammin kuin
esimerkiksi ajonaikaista käyttäytymistä testaavat \textit{unit testit}.
Työn viidennen luvun viimeisessä esimerkissä esitelty get-kutsu on yksinkertainen
ja helppolukuinen, mutta jos muuttujan \inlinecode{object} tyyppiä myöhemmin
muutetaan toiseen muotoon muistamatta päivittää myös get-kutsua, virheellisestä
get-kutsusta voi muodostua ohjelmaan hankala bugi.

Flow, TypeScript ja Closure sallivat sekä dynaamisesti että
staattisesti tyypitetyn koodin käytön ja hämärtävät rajaa niiden välillä.
Jää ohjelmoijan päätettäväksi onko yllä mainitun get-kutsun yksinkertaisuus
tyyppiturvattomuuden arvoista. JavaScript-kehityksessä hyväksi todetut
suunnittelumallit ovat edelleen käytettävissä tyyppiannotoidussakin
koodissa, vaikka välillä tyyppiturvallisuutta olisikin sen vuoksi höllättävä.
Lopputulos ei ole ainakaan sen turvattomampaa kuin normaali JavaScript-koodikaan
ja tyyppiturvaton osa koodia voidaan usein rajata pieneksi, muilla tavoin
testatuksi osa-alueekseen muuten staattisesti tarkastetussa kokonaisuudessa.
Samankaltaisuus ja yhteensopivuus tavallisen JavaScriptin kanssa onkin
näiden työkalujen tärkein etu muihin staattisesti tyypitettyihin kieliin
nähden. JavaScript projektit on mahdollista muuttaa vaiheittain staattisesti
tyypitetyksi kirjoittamatta koko ohjelman koodia alusta asti uudestaan, eikä
ohjelmoijan itsensä tarvitse korvata kaikkea JavaScriptista oppimaansa
uuden kielen tavoilla ja yhteisöllä. Closuren JSDoc-tyyliset kommentit
saattavat jo ollakin JavaScript-ohjelmoijalle tuttuja, sillä niitä käytetään
usein koodin dokumentointiin vaikka tyyppien oikeellisuutta ei
tarkastettaisikaan Closurella. Myös TypeScript- ja Flow-annotaatiot voivat
helpottaa uusien JavaScript-ohjelmoijien tutustumista uuteen projektiin
tuomallaan dokumentaatioarvolla.

Samankaltaisuus ja yhteensopivuus jo ennestään suositun JavaScriptin kanssa
yhdistettynä tyyppiturvallisuuteen, koodin kirjoittamista helpottaviin
työkaluihin ja selkeämmin dokumentoituun koodiin ovat nostaneet erityisesti
TypeScriptin käytön nopeaan kasvuun. Vuosittaisessa JavaScript-yhteisölle
tarkoitetussa \textit{State Of JS} kyselytutkimuksessa 46.7\% vuonna 2018
vastanneista ilmoitti käyttäneensä TypeScriptiä ja haluavansa
käyttää sitä uudestaan \cite{StateOfJs2018}.
Vielä vuonna 2016 luku oli 21\% \cite{StateOfJs2016}. TypeScript ja
JavaScriptin staattinen tyypitys näyttävät kasvattavan suosiotaan myös
tulevaisuudessa. Vuoden 2018 State Of JS kyselyyn vastanneista 33.7\%
vastasi haluavansa\newline
opetella TypeScriptiä ja 34.5\% Flow'ta.
