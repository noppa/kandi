\chapter{Virheiden havaitseminen}

\section{Virheiden havaitseminen}

Kenties tärkein staattisen tyyppijärjestelmän tehtävä on havaita ja estää
ohjelmoijan virheitä. Tässä esitellyt työkalut, mahdollisesti Closure
kääntäjää lukuunottamatta, onkin kehitetty erityisesti tätä tarkoitusta
varten.

Kaikki kolme työkalua antaisivat käännösvirheen jos esimerkeissä
\ref{lst:ostoskorin_hinta_clojure} ja \ref{lst:ostoskorin_hinta_flow}
esiteltyä funktiota kutsuttaisiin virheellisesti esimerkiksi listalla
hintaa kuvaavia numeroita, sillä funktion parametrin on annotoitu olevan
lista ``Ostos''-tyyppimääritelmän mukaisia objekteja. Esimerkiksi
virheellinen kutsu
\colorbox{lightgray}{\lstinline|ostoskorinHinta([5, 10, 15])|} ei itse
asiassa aiheuttaisi suoritettaessa ohjelman keskeyttävää virhettä.
\colorbox{lightgray}{\lstinline|ostos.hinta|} ilmaisu on sallittu vaikka
muuttuja \colorbox{lightgray}{\lstinline|ostos|} olisikin arvoltaan numero
eikä objekti. Tällöin ilmaisun arvo on \colorbox{lightgray}{\lstinline|undefined|}
ja lausekkeen \colorbox{lightgray}{\lstinline|summa += ostos.hinta|} jälkeen
\colorbox{lightgray}{\lstinline|summa|} muuttujan arvo on erityinen
ei-numeroa kuvaava \colorbox{lightgray}{\lstinline|NaN|} \cite{Ecma262NaN}.
Käännösaikaisen tarkistamisen merkitys korostuu erityisen hyödylliseksi
tämänkaltaisen ohjelmointivirheen kohdalla, sillä virhe ei välttämättä ole
muutoin helposti havaittavissa. Funktiokutsu ei aiheuttaisi helposti
todennettavaa suoritusaikaista virhettä, joten ei-toivottu palautusarvo
\colorbox{lightgray}{\lstinline|NaN|} saattaisi kiertää ohjelman
operaatioiden välillä pitkällekin aiheuttaen muita loogisia virheitä.

\section{Ohjelman optimointi käännösvaiheessa}
TODO

\section{Tyyppimäärittelyt dokumentaationa}
TODO